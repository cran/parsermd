\documentclass{beamer}\usepackage[]{graphicx}\usepackage[]{xcolor}
% maxwidth is the original width if it is less than linewidth
% otherwise use linewidth (to make sure the graphics do not exceed the margin)
\makeatletter
\def\maxwidth{ %
  \ifdim\Gin@nat@width>\linewidth
    \linewidth
  \else
    \Gin@nat@width
  \fi
}
\makeatother

\definecolor{fgcolor}{rgb}{0.345, 0.345, 0.345}
\newcommand{\hlnum}[1]{\textcolor[rgb]{0.686,0.059,0.569}{#1}}%
\newcommand{\hlsng}[1]{\textcolor[rgb]{0.192,0.494,0.8}{#1}}%
\newcommand{\hlcom}[1]{\textcolor[rgb]{0.678,0.584,0.686}{\textit{#1}}}%
\newcommand{\hlopt}[1]{\textcolor[rgb]{0,0,0}{#1}}%
\newcommand{\hldef}[1]{\textcolor[rgb]{0.345,0.345,0.345}{#1}}%
\newcommand{\hlkwa}[1]{\textcolor[rgb]{0.161,0.373,0.58}{\textbf{#1}}}%
\newcommand{\hlkwb}[1]{\textcolor[rgb]{0.69,0.353,0.396}{#1}}%
\newcommand{\hlkwc}[1]{\textcolor[rgb]{0.333,0.667,0.333}{#1}}%
\newcommand{\hlkwd}[1]{\textcolor[rgb]{0.737,0.353,0.396}{\textbf{#1}}}%
\let\hlipl\hlkwb

\usepackage{framed}
\makeatletter
\newenvironment{kframe}{%
 \def\at@end@of@kframe{}%
 \ifinner\ifhmode%
  \def\at@end@of@kframe{\end{minipage}}%
  \begin{minipage}{\columnwidth}%
 \fi\fi%
 \def\FrameCommand##1{\hskip\@totalleftmargin \hskip-\fboxsep
 \colorbox{shadecolor}{##1}\hskip-\fboxsep
     % There is no \\@totalrightmargin, so:
     \hskip-\linewidth \hskip-\@totalleftmargin \hskip\columnwidth}%
 \MakeFramed {\advance\hsize-\width
   \@totalleftmargin\z@ \linewidth\hsize
   \@setminipage}}%
 {\par\unskip\endMakeFramed%
 \at@end@of@kframe}
\makeatother

\definecolor{shadecolor}{rgb}{.97, .97, .97}
\definecolor{messagecolor}{rgb}{0, 0, 0}
\definecolor{warningcolor}{rgb}{1, 0, 1}
\definecolor{errorcolor}{rgb}{1, 0, 0}
\newenvironment{knitrout}{}{} % an empty environment to be redefined in TeX

\usepackage{alltt}
\IfFileExists{upquote.sty}{\usepackage{upquote}}{}
\begin{document}

\title{Show Plots Step by Step with knitr in Beamer}
\author{Yihui Xie}

\maketitle

\begin{frame}
The question came from \url{http://r.789695.n4.nabble.com/plots-for-presentation-td4645869.html}
\end{frame}

% very important to use option [fragile] for frames containing code output!
\begin{frame}[fragile]
We will add ``\textbackslash{}only'' to the plots so they show up one by one.

\begin{knitrout}\footnotesize
\definecolor{shadecolor}{rgb}{0.969, 0.969, 0.969}\color{fgcolor}\begin{kframe}
\begin{alltt}
\hlkwd{par}\hldef{(}\hlkwc{mar} \hldef{=} \hlkwd{c}\hldef{(}\hlnum{4}\hldef{,} \hlnum{4}\hldef{,} \hlnum{0.1}\hldef{,} \hlnum{0.1}\hldef{))}
\hldef{x} \hlkwb{=} \hlkwd{rnorm}\hldef{(}\hlnum{100}\hldef{)}
\hlkwd{hist}\hldef{(x,} \hlkwc{main} \hldef{=} \hlsng{""}\hldef{)}
\hlkwd{rug}\hldef{(x)}
\hlkwd{abline}\hldef{(}\hlkwc{v} \hldef{=} \hlkwd{mean}\hldef{(x),} \hlkwc{lty} \hldef{=} \hlnum{2}\hldef{,} \hlkwc{lwd} \hldef{=} \hlnum{3}\hldef{)}
\end{alltt}
\end{kframe}

{\centering \only<1>{\includegraphics[width=.5\linewidth]{figure/053-beamer-only-boring-plots-1}} 
\only<2>{\includegraphics[width=.5\linewidth]{figure/053-beamer-only-boring-plots-2}} 
\only<3>{\includegraphics[width=.5\linewidth]{figure/053-beamer-only-boring-plots-3}} 

}


\end{knitrout}
\end{frame}

\end{document}
